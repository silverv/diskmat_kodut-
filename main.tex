\documentclass{article}
\usepackage[utf8]{inputenc}
\usepackage[estonian]{babel}
\usepackage{amsmath}
\title{Diskreetse matemaatika kodutöö}
\author{Silver Valdvee}
\date{detsember 2017}
\usepackage{tikz}
\usetikzlibrary{matrix,calc}

%isolated term
%#1 - Optional. Space between node and grouping line. Default=0
%#2 - node
%#3 - filling color
\newcommand{\implicantsol}[3][0]{
    \draw[rounded corners=3pt, fill=#3, opacity=0.3] ($(#2.north west)+(135:#1)$) rectangle ($(#2.south east)+(-45:#1)$);
    }


%internal group
%#1 - Optional. Space between node and grouping line. Default=0
%#2 - top left node
%#3 - bottom right node
%#4 - filling color
\newcommand{\implicant}[4][0]{
    \draw[rounded corners=3pt, fill=#4, opacity=0.3] ($(#2.north west)+(135:#1)$) rectangle ($(#3.south east)+(-45:#1)$);
    }

%group lateral borders
%#1 - Optional. Space between node and grouping line. Default=0
%#2 - top left node
%#3 - bottom right node
%#4 - filling color
\newcommand{\implicantcostats}[4][0]{
    \draw[rounded corners=3pt, fill=#4, opacity=0.3] ($(rf.east |- #2.north)+(90:#1)$)-| ($(#2.east)+(0:#1)$) |- ($(rf.east |- #3.south)+(-90:#1)$);
    \draw[rounded corners=3pt, fill=#4, opacity=0.3] ($(cf.west |- #2.north)+(90:#1)$) -| ($(#3.west)+(180:#1)$) |- ($(cf.west |- #3.south)+(-90:#1)$);
}

%group top-bottom borders
%#1 - Optional. Space between node and grouping line. Default=0
%#2 - top left node
%#3 - bottom right node
%#4 - filling color
\newcommand{\implicantdaltbaix}[4][0]{
    \draw[rounded corners=3pt, fill=#4, opacity=0.3] ($(cf.south -| #2.west)+(180:#1)$) |- ($(#2.south)+(-90:#1)$) -| ($(cf.south -| #3.east)+(0:#1)$);
    \draw[rounded corners=3pt, fill=#4, opacity=0.3] ($(rf.north -| #2.west)+(180:#1)$) |- ($(#3.north)+(90:#1)$) -| ($(rf.north -| #3.east)+(0:#1)$);
}

%group corners
%#1 - Optional. Space between node and grouping line. Default=0
%#2 - filling color
\newcommand{\implicantcantons}[2][0]{
    \draw[rounded corners=3pt, opacity=.3] ($(rf.east |- 0.south)+(-90:#1)$) -| ($(0.east |- cf.south)+(0:#1)$);
    \draw[rounded corners=3pt, opacity=.3] ($(rf.east |- 8.north)+(90:#1)$) -| ($(8.east |- rf.north)+(0:#1)$);
    \draw[rounded corners=3pt, opacity=.3] ($(cf.west |- 2.south)+(-90:#1)$) -| ($(2.west |- cf.south)+(180:#1)$);
    \draw[rounded corners=3pt, opacity=.3] ($(cf.west |- 10.north)+(90:#1)$) -| ($(10.west |- rf.north)+(180:#1)$);
    \fill[rounded corners=3pt, fill=#2, opacity=.3] ($(rf.east |- 0.south)+(-90:#1)$) -|  ($(0.east |- cf.south)+(0:#1)$) [sharp corners] ($(rf.east |- 0.south)+(-90:#1)$) |-  ($(0.east |- cf.south)+(0:#1)$) ;
    \fill[rounded corners=3pt, fill=#2, opacity=.3] ($(rf.east |- 8.north)+(90:#1)$) -| ($(8.east |- rf.north)+(0:#1)$) [sharp corners] ($(rf.east |- 8.north)+(90:#1)$) |- ($(8.east |- rf.north)+(0:#1)$) ;
    \fill[rounded corners=3pt, fill=#2, opacity=.3] ($(cf.west |- 2.south)+(-90:#1)$) -| ($(2.west |- cf.south)+(180:#1)$) [sharp corners]($(cf.west |- 2.south)+(-90:#1)$) |- ($(2.west |- cf.south)+(180:#1)$) ;
    \fill[rounded corners=3pt, fill=#2, opacity=.3] ($(cf.west |- 10.north)+(90:#1)$) -| ($(10.west |- rf.north)+(180:#1)$) [sharp corners] ($(cf.west |- 10.north)+(90:#1)$) |- ($(10.west |- rf.north)+(180:#1)$) ;
}

%Empty Karnaugh map 4x4
\newenvironment{Karnaugh}%
{
\begin{tikzpicture}[baseline=(current bounding box.north),scale=0.8]
\draw (0,0) grid (4,4);
\draw (0,4) -- node [pos=0.9,above right,anchor=south west] {$x_3x_4$} node [pos=0.9,below left,anchor=north east] {$x_1x_2$} ++(135:1);
%
\matrix (mapa) [matrix of nodes,
        column sep={0.8cm,between origins},
        row sep={0.8cm,between origins},
        every node/.style={minimum size=0.3mm},
        anchor=8.center,
        ampersand replacement=\&] at (0.5,0.5)
{
                       \& |(c00)| 00         \& |(c01)| 01         \& |(c11)| 11         \& |(c10)| 10         \& |(cf)| \phantom{00} \\
|(r00)| 00             \& |(0)|  \phantom{0} \& |(1)|  \phantom{0} \& |(3)|  \phantom{0} \& |(2)|  \phantom{0} \&                     \\
|(r01)| 01             \& |(4)|  \phantom{0} \& |(5)|  \phantom{0} \& |(7)|  \phantom{0} \& |(6)|  \phantom{0} \&                     \\
|(r11)| 11             \& |(12)| \phantom{0} \& |(13)| \phantom{0} \& |(15)| \phantom{0} \& |(14)| \phantom{0} \&                     \\
|(r10)| 10             \& |(8)|  \phantom{0} \& |(9)|  \phantom{0} \& |(11)| \phantom{0} \& |(10)| \phantom{0} \&                     \\
|(rf) | \phantom{00}   \&                    \&                    \&                    \&                    \&                     \\
};
}%
{
\end{tikzpicture}
}

%Empty Karnaugh map 2x4
\newenvironment{Karnaughvuit}%
{
\begin{tikzpicture}[baseline=(current bounding box.north),scale=0.8]
\draw (0,0) grid (4,2);
\draw (0,2) -- node [pos=0.7,above right,anchor=south west] {bc} node [pos=0.7,below left,anchor=north east] {a} ++(135:1);
%
\matrix (mapa) [matrix of nodes,
        column sep={0.8cm,between origins},
        row sep={0.8cm,between origins},
        every node/.style={minimum size=0.3mm},
        anchor=4.center,
        ampersand replacement=\&] at (0.5,0.5)
{
                      \& |(c00)| 00         \& |(c01)| 01         \& |(c11)| 11         \& |(c10)| 10         \& |(cf)| \phantom{00} \\
|(r00)| 0             \& |(0)|  \phantom{0} \& |(1)|  \phantom{0} \& |(3)|  \phantom{0} \& |(2)|  \phantom{0} \&                     \\
|(r01)| 1             \& |(4)|  \phantom{0} \& |(5)|  \phantom{0} \& |(7)|  \phantom{0} \& |(6)|  \phantom{0} \&                     \\
|(rf) | \phantom{00}  \&                    \&                    \&                    \&                    \&                     \\
};
}%
{
\end{tikzpicture}
}

%Empty Karnaugh map 2x2
\newenvironment{Karnaughquatre}%
{
\begin{tikzpicture}[baseline=(current bounding box.north),scale=0.8]
\draw (0,0) grid (2,2);
\draw (0,2) -- node [pos=0.7,above right,anchor=south west] {b} node [pos=0.7,below left,anchor=north east] {a} ++(135:1);
%
\matrix (mapa) [matrix of nodes,
        column sep={0.8cm,between origins},
        row sep={0.8cm,between origins},
        every node/.style={minimum size=0.3mm},
        anchor=2.center,
        ampersand replacement=\&] at (0.5,0.5)
{
          \& |(c00)| 0          \& |(c01)| 1  \\
|(r00)| 0 \& |(0)|  \phantom{0} \& |(1)|  \phantom{0} \\
|(r01)| 1 \& |(2)|  \phantom{0} \& |(3)|  \phantom{0} \\
};
}%
{
\end{tikzpicture}
}

%Defines 8 or 16 values (0,1,X)
\newcommand{\contingut}[1]{%
\foreach \x [count=\xi from 0]  in {#1}
     \path (\xi) node {\x};
}

%Places 1 in listed positions
\newcommand{\minterms}[1]{%
    \foreach \x in {#1}
        \path (\x) node {1};
}

%Places 0 in listed positions
\newcommand{\maxterms}[1]{%
    \foreach \x in {#1}
        \path (\x) node {0};
}

%Places X in listed positions
\newcommand{\indeterminats}[1]{%
    \foreach \x in {#1}
        \path (\x) node {X};
}

\begin{document}

\maketitle

\section{Funktsiooni tekitamine}
\begin{enumerate}
    \item Martiklinumber: 179390
    \item HEX: 2BCBE
    \item Seitsmega korrutades esimene seitsmekohaline 16. arv: 3AAE292
    \item Unikaalsed järguväärtused määravad 1-de piirkonna: 3, A, E, 2, 9
    \item Unikaalsed järguväärtused 10. süsteemis 1-de piirkond: 3, 10, 14, 2, 9
    \item Eelkirjeldatud viisil saadud ja hetkel kalkulaatoris näidatava 16. arvu korrutis 7 kuni 9-kohalise arvu saamiseni: $4E9F5919E$
    \item 9-kohalise tekkinud 16ndarvu need järguväärtused 0 ... 15, mis ei kuulu juba 1-de piirkonda, moodustavad funktsiooni määramatuspiirkonna: $4, F, 5, 1 \rightarrow 4, 15, 5, 1 \rightarrow 1, 4, 5, 15$
    \item Ülejäänud arvud vahemikus 0 ...15 (mis puuduvad nii 1de piirkonnas kui ka määramatuspiirkonnas) moodustavad 0de piirkonna: 0, 6, 7, 8, 11, 12, 13
\end{enumerate}
Saan funktsiooni:
\[f(x_1...x_4)=\Sigma(2,3,9,10,14)_1(0,6,7,8,11,12,13)_0(1,4,5,15)_{\_}
\]
Teen sellele tõeväärtustabeli:
\begin{table}[]
\centering
\caption{tõeväärtustabel funktsioonile}
\label{my-label}
\begin{tabular}{|c|c|c|c|c||c|}
\hline
$x_4$ & $x_3$ & $x_2$ & $x_1$ & 10. arv & F \\ \hline\hline
0  & 0  & 0  & 0  & 0       & 0 \\ \hline
0  & 0  & 0  & 1  & 1       & - \\ \hline
0  & 0  & 1  & 0  & 2       & 1 \\ \hline
0  & 0  & 1  & 1  & 3       & 1 \\ \hline
0  & 1  & 0  & 0  & 4       & - \\ \hline
0  & 1  & 0  & 1  & 5       & - \\ \hline
0  & 1  & 1  & 0  & 6       & 0 \\ \hline
0  & 1  & 1  & 1  & 7       & 0 \\ \hline
1  & 0  & 0  & 0  & 8       & 0 \\ \hline
1  & 0  & 0  & 1  & 9       & 1 \\ \hline
1  & 0  & 1  & 0  & 10      & 1 \\ \hline
1  & 0  & 1  & 1  & 11      & 0 \\ \hline
1  & 1  & 0  & 0  & 12      & 0 \\ \hline
1  & 1  & 0  & 1  & 13      & 0 \\ \hline
1  & 1  & 1  & 0  & 14      & 1 \\ \hline
1  & 1  & 1  & 1  & 15      & - \\ \hline
\end{tabular}
\end{table}
McCluskey meetod:
\begin{table}[]
\centering
\caption{McCluskey meetod}
\label{my-label}
\begin{tabular}{|l|l|l|l|}
\hline
indeks & 1. pk & 2-sed interv. & 4-sed \\ \hline\hline
0 & 0000 x & \begin{tabular}[c]{@{}c@{}}000- x\\ 0-00 x\\ -000 x\end{tabular} & \begin{tabular}[c]{@{}c@{}}0-0- A3\\ - -00 A4\end{tabular} \\ \hline
1 & \begin{tabular}[c]{@{}c@{}}0001 x\\ 0100 x\\ 1000 x\end{tabular} & \begin{tabular}[c]{@{}c@{}}0-01 x\\ 010- xx\\ 01-0 x\\ -100 xx\\ 1-00 A1\end{tabular} & \begin{tabular}[c]{@{}c@{}}-10- A5\\ 01- - A6\end{tabular} \\ \hline
2 & \begin{tabular}[c]{@{}c@{}}0101 x\\ 0110 x\\ 1100 x\end{tabular} & \begin{tabular}[c]{@{}c@{}}01-1 x\\ -101 xx\\ 011- x\\ 110- x\end{tabular} & -1-1 A7 \\ \hline
3 & \begin{tabular}[c]{@{}c@{}}0111 x\\ 1011 x\\ 1101 x\end{tabular} & \begin{tabular}[c]{@{}c@{}}-111 x\\ 1-11 A2\\ 11-1 x\end{tabular} &  \\ \hline
4 & 1111 x &  &  \\ \hline
\end{tabular}
\end{table}
\begin{figure}[!h]
\centering
\begin{Karnaugh}
    \contingut{0, -, 1, 1,
    0/-, 0/-, 0, 0,
    0, 1, 1, 0,
    0, 0, 1, 0/-}
   \implicant{15}{11}{yellow}
   \implicant{4}{6}{yellow}
   \implicant{4}{13}{yellow}
   \implicant{0}{8}{yellow}
\end{Karnaugh}
\caption{Karnaugh kaart laiendatud 0de piirkonnaga}
\label{fig:figure2}
\end{figure}

\begin{figure}[!h]
\centering
\begin{Karnaugh}
    \contingut{0, -/1, 1, 1,
    -, -, 0, 0,
    0, 1, 1, 0,
    0, 0, 1, -}
    \implicantdaltbaix[3pt]{1}{9}{yellow}
   \implicant{14}{10}{yellow}
   \implicant{3}{2}{yellow}
\end{Karnaugh}
\caption{Karnaugh kaart laiendatud 1de piirkonnaga}
\label{fig:figure2}
\end{figure}
\end{document}
