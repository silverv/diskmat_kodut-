\documentclass[12pt]{article}
\usepackage[utf8]{inputenc}
\usepackage[estonian]{babel}
\usepackage{float}
\usepackage[fleqn]{amsmath}
\usepackage[a4paper, margin=0.5in, bottom=0.9in, top=0.9in]{geometry}
\usepackage[makeroom]{cancel}
\usepackage{titling}
\sloppy
\newlength{\overwritelength}
\newlength{\minimumoverwritelength}
\setlength{\minimumoverwritelength}{1cm}
\newcommand{\overwrite}[3][BurntOrange]{%
  \settowidth{\overwritelength}{$#2$}%
  \ifdim\overwritelength<\minimumoverwritelength%
    \setlength{\overwritelength}{\minimumoverwritelength}\fi%
  \stackrel
    {%
      \begin{minipage}{\overwritelength}%
        \color{#1}\centering\small #3\\%
        \rule{1pt}{9pt}%
      \end{minipage}}
    {\colorbox{#1!50}{\color{black}$\displaystyle#2$}}}
\usepackage{xcolor, colortbl, soul}
\setul{0.5ex}{0.3ex}
\definecolor{LightCyan}{rgb}{0.88,1,1}
\newcolumntype{g}{>{\columncolor{LightCyan}}c}

% Widebar is now a replacement for widebar
\DeclareFontFamily{U}{mathx}{\hyphenchar\font45}
\DeclareFontShape{U}{mathx}{m}{n}{ <-> mathx10 }{}
\DeclareSymbolFont{mathx}{U}{mathx}{m}{n}
\DeclareFontSubstitution{U}{mathx}{m}{n}
\DeclareMathAccent{\widebar}{\mathalpha}{mathx}{"73}

\makeatletter
\newcommand{\cwidebar}[2][0]{{\mathpalette\@cwidebar{{#1}{#2}}}}
\newcommand{\@cwidebar}[2]{\@cwideb@r{#1}#2}
\newcommand{\@cwideb@r}[3]{%
  \sbox\z@{$\m@th#1\mkern-#2mu#3\mkern#2mu$}%
  \widebar{\box\z@}%
}
\makeatother
\newcommand{\wb}{\widebar}

\newsavebox\MBox

\newcommand\Cline[2][red]{{\sbox\MBox{$#2$}%
  \rlap{\usebox\MBox}\color{#1}\rule[-1.2\dp\MBox]{\wd\MBox}{0.5pt}}}
  
\setlength{\parindent}{0pt}


\usepackage{tikz}
\usetikzlibrary{calc}
\usetikzlibrary{decorations.text}
\newcommand{\tikzmark}[1]{\tikz[overlay,remember picture] \node (#1) {};}
\usepackage{tikz}
\usetikzlibrary{matrix,calc}

%isolated term
%#1 - Optional. Space between node and grouping line. Default=0
%#2 - node
%#3 - filling color
\newcommand{\implicantsol}[3][0]{
    \draw[rounded corners=3pt, fill=#3, opacity=0.3] ($(#2.north west)+(135:#1)$) rectangle ($(#2.south east)+(-45:#1)$);
    }


%internal group
%#1 - Optional. Space between node and grouping line. Default=0
%#2 - top left node
%#3 - bottom right node
%#4 - filling color
\newcommand{\implicant}[4][0]{
    \draw[rounded corners=3pt, fill=#4, opacity=0.3] ($(#2.north west)+(135:#1)$) rectangle ($(#3.south east)+(-45:#1)$);
    }

%group lateral borders
%#1 - Optional. Space between node and grouping line. Default=0
%#2 - top left node
%#3 - bottom right node
%#4 - filling color
\newcommand{\implicantcostats}[4][0]{
    \draw[rounded corners=3pt, fill=#4, opacity=0.3] ($(rf.east |- #2.north)+(90:#1)$)-| ($(#2.east)+(0:#1)$) |- ($(rf.east |- #3.south)+(-90:#1)$);
    \draw[rounded corners=3pt, fill=#4, opacity=0.3] ($(cf.west |- #2.north)+(90:#1)$) -| ($(#3.west)+(180:#1)$) |- ($(cf.west |- #3.south)+(-90:#1)$);
}

%group top-bottom borders
%#1 - Optional. Space between node and grouping line. Default=0
%#2 - top left node
%#3 - bottom right node
%#4 - filling color
\newcommand{\implicantdaltbaix}[4][0]{
    \draw[rounded corners=3pt, fill=#4, opacity=0.3] ($(cf.south -| #2.west)+(180:#1)$) |- ($(#2.south)+(-90:#1)$) -| ($(cf.south -| #3.east)+(0:#1)$);
    \draw[rounded corners=3pt, fill=#4, opacity=0.3] ($(rf.north -| #2.west)+(180:#1)$) |- ($(#3.north)+(90:#1)$) -| ($(rf.north -| #3.east)+(0:#1)$);
}

%group corners
%#1 - Optional. Space between node and grouping line. Default=0
%#2 - filling color
\newcommand{\implicantcantons}[2][0]{
    \draw[rounded corners=3pt, opacity=.3] ($(rf.east |- 0.south)+(-90:#1)$) -| ($(0.east |- cf.south)+(0:#1)$);
    \draw[rounded corners=3pt, opacity=.3] ($(rf.east |- 8.north)+(90:#1)$) -| ($(8.east |- rf.north)+(0:#1)$);
    \draw[rounded corners=3pt, opacity=.3] ($(cf.west |- 2.south)+(-90:#1)$) -| ($(2.west |- cf.south)+(180:#1)$);
    \draw[rounded corners=3pt, opacity=.3] ($(cf.west |- 10.north)+(90:#1)$) -| ($(10.west |- rf.north)+(180:#1)$);
    \fill[rounded corners=3pt, fill=#2, opacity=.3] ($(rf.east |- 0.south)+(-90:#1)$) -|  ($(0.east |- cf.south)+(0:#1)$) [sharp corners] ($(rf.east |- 0.south)+(-90:#1)$) |-  ($(0.east |- cf.south)+(0:#1)$) ;
    \fill[rounded corners=3pt, fill=#2, opacity=.3] ($(rf.east |- 8.north)+(90:#1)$) -| ($(8.east |- rf.north)+(0:#1)$) [sharp corners] ($(rf.east |- 8.north)+(90:#1)$) |- ($(8.east |- rf.north)+(0:#1)$) ;
    \fill[rounded corners=3pt, fill=#2, opacity=.3] ($(cf.west |- 2.south)+(-90:#1)$) -| ($(2.west |- cf.south)+(180:#1)$) [sharp corners]($(cf.west |- 2.south)+(-90:#1)$) |- ($(2.west |- cf.south)+(180:#1)$) ;
    \fill[rounded corners=3pt, fill=#2, opacity=.3] ($(cf.west |- 10.north)+(90:#1)$) -| ($(10.west |- rf.north)+(180:#1)$) [sharp corners] ($(cf.west |- 10.north)+(90:#1)$) |- ($(10.west |- rf.north)+(180:#1)$) ;
}

%Empty Karnaugh map 4x4
\newenvironment{Karnaugh}%
{
\begin{tikzpicture}[baseline=(current bounding box.north),scale=0.8]
\draw (0,0) grid (4,4);
\draw (0,4) -- node [pos=0.9,above right,anchor=south west] {$x_3x_4$} node [pos=0.9,below left,anchor=north east] {$x_1x_2$} ++(135:1);
%
\matrix (mapa) [matrix of nodes,
        column sep={0.8cm,between origins},
        row sep={0.8cm,between origins},
        every node/.style={minimum size=0.3mm},
        anchor=8.center,
        ampersand replacement=\&] at (0.5,0.5)
{
                       \& |(c00)| 00         \& |(c01)| 01         \& |(c11)| 11         \& |(c10)| 10         \& |(cf)| \phantom{00} \\
|(r00)| 00             \& |(0)|  \phantom{0} \& |(1)|  \phantom{0} \& |(3)|  \phantom{0} \& |(2)|  \phantom{0} \&                     \\
|(r01)| 01             \& |(4)|  \phantom{0} \& |(5)|  \phantom{0} \& |(7)|  \phantom{0} \& |(6)|  \phantom{0} \&                     \\
|(r11)| 11             \& |(12)| \phantom{0} \& |(13)| \phantom{0} \& |(15)| \phantom{0} \& |(14)| \phantom{0} \&                     \\
|(r10)| 10             \& |(8)|  \phantom{0} \& |(9)|  \phantom{0} \& |(11)| \phantom{0} \& |(10)| \phantom{0} \&                     \\
|(rf) | \phantom{00}   \&                    \&                    \&                    \&                    \&                     \\
};
}%
{
\end{tikzpicture}
}

%Empty Karnaugh map 2x4
\newenvironment{Karnaughvuit}%
{
\begin{tikzpicture}[baseline=(current bounding box.north),scale=0.8]
\draw (0,0) grid (4,2);
\draw (0,2) -- node [pos=0.7,above right,anchor=south west] {bc} node [pos=0.7,below left,anchor=north east] {a} ++(135:1);
%
\matrix (mapa) [matrix of nodes,
        column sep={0.8cm,between origins},
        row sep={0.8cm,between origins},
        every node/.style={minimum size=0.3mm},
        anchor=4.center,
        ampersand replacement=\&] at (0.5,0.5)
{
                      \& |(c00)| 00         \& |(c01)| 01         \& |(c11)| 11         \& |(c10)| 10         \& |(cf)| \phantom{00} \\
|(r00)| 0             \& |(0)|  \phantom{0} \& |(1)|  \phantom{0} \& |(3)|  \phantom{0} \& |(2)|  \phantom{0} \&                     \\
|(r01)| 1             \& |(4)|  \phantom{0} \& |(5)|  \phantom{0} \& |(7)|  \phantom{0} \& |(6)|  \phantom{0} \&                     \\
|(rf) | \phantom{00}  \&                    \&                    \&                    \&                    \&                     \\
};
}%
{
\end{tikzpicture}
}

%Empty Karnaugh map 2x2
\newenvironment{Karnaughquatre}%
{
\begin{tikzpicture}[baseline=(current bounding box.north),scale=0.8]
\draw (0,0) grid (2,2);
\draw (0,2) -- node [pos=0.7,above right,anchor=south west] {b} node [pos=0.7,below left,anchor=north east] {a} ++(135:1);
%
\matrix (mapa) [matrix of nodes,
        column sep={0.8cm,between origins},
        row sep={0.8cm,between origins},
        every node/.style={minimum size=0.3mm},
        anchor=2.center,
        ampersand replacement=\&] at (0.5,0.5)
{
          \& |(c00)| 0          \& |(c01)| 1  \\
|(r00)| 0 \& |(0)|  \phantom{0} \& |(1)|  \phantom{0} \\
|(r01)| 1 \& |(2)|  \phantom{0} \& |(3)|  \phantom{0} \\
};
}%
{
\end{tikzpicture}
}

%Defines 8 or 16 values (0,1,X)
\newcommand{\contingut}[1]{%
\foreach \x [count=\xi from 0]  in {#1}
     \path (\xi) node {\x};
}

%Places 1 in listed positions
\newcommand{\minterms}[1]{%
    \foreach \x in {#1}
        \path (\x) node {1};
}

%Places 0 in listed positions
\newcommand{\maxterms}[1]{%
    \foreach \x in {#1}
        \path (\x) node {0};
}

%Places X in listed positions
\newcommand{\indeterminats}[1]{%
    \foreach \x in {#1}
        \path (\x) node {X};
}


\usepackage{fancyhdr, lastpage}
\fancypagestyle{firstpage}{%
    \fancyhf{} % sets both header and footer to nothing
    \renewcommand{\headrulewidth}{0pt}
  \chead{Tallinna Tehnikaülikool}
  \cfoot{\today}
}
\fancypagestyle{plain}{
  \fancyhf{}% Clear header/footer
  \lhead{Diskreetse matemaatika kodutöö}
  \fancyfoot[L]{Silver Valdvee}% Left footer
  \fancyfoot[R]{\thepage\  / \pageref{LastPage}}% Right footer
}
\pagestyle{plain}% Set page style to plain.
\setlength{\jot}{20pt}
\begin{document}
\begin{titlepage}
\centering
\vspace*{\fill}
\Huge{Diskreetse matemaatika kodutöö}
\begin{flushright}
\large{
Silver Valdvee\\
IAIB179390\\
IAIB14}
\end{flushright}
\vspace*{\fill}
\thispagestyle{firstpage}
\end{titlepage}
\section{Funktsiooni tekitamine}
\begin{enumerate}
    \item Martiklinumber: 179390
    \item HEX: 2BCBE
    \item Seitsmega korrutades esimene seitsmekohaline 16. arv: 3AAE292
    \item Unikaalsed järguväärtused määravad 1-de piirkonna: 3, A, E, 2, 9
    \item Unikaalsed järguväärtused 10. süsteemis 1-de piirkond: 3, 10, 14, 2, 9
    \item Eelkirjeldatud viisil saadud ja hetkel kalkulaatoris näidatava 16. arvu korrutis 7 kuni 9-kohalise arvu saamiseni: $4E9F5919E$
    \item 9-kohalise tekkinud 16ndarvu need järguväärtused 0 ... 15, mis ei kuulu juba 1-de piirkonda, moodustavad funktsiooni määramatuspiirkonna: $4, F, 5, 1 \rightarrow 4, 15, 5, 1 \rightarrow 1, 4, 5, 15$
    \item Ülejäänud arvud vahemikus 0 ...15 (mis puuduvad nii 1de piirkonnas kui ka määramatuspiirkonnas) moodustavad 0de piirkonna: 0, 6, 7, 8, 11, 12, 13
\end{enumerate}
Saan funktsiooni:
\[f(x_1...x_4)=\Sigma(2,3,9,10,14)_1(0,6,7,8,11,12,13)_0(1,4,5,15)_{\_}
\]
\section{Tõeväärtustabel}
Teen funktsioonile tõeväärtustabeli:
\begin{table}[H]
\centering
\caption{tõeväärtustabel funktsioonile}
\label{truth-table}
\begin{tabular}{|c|c|c|c|c||c|}
\hline
$x_1$ & $x_2$ & $x_3$ & $x_4$ & 10. arv & F \\ \hline\hline
0  & 0  & 0  & 0  & 0       & 0 \\ \hline
0  & 0  & 0  & 1  & 1       & - \\ \hline
0  & 0  & 1  & 0  & 2       & 1 \\ \hline
0  & 0  & 1  & 1  & 3       & 1 \\ \hline
0  & 1  & 0  & 0  & 4       & - \\ \hline
0  & 1  & 0  & 1  & 5       & - \\ \hline
0  & 1  & 1  & 0  & 6       & 0 \\ \hline
0  & 1  & 1  & 1  & 7       & 0 \\ \hline
1  & 0  & 0  & 0  & 8       & 0 \\ \hline
1  & 0  & 0  & 1  & 9       & 1 \\ \hline
1  & 0  & 1  & 0  & 10      & 1 \\ \hline
1  & 0  & 1  & 1  & 11      & 0 \\ \hline
1  & 1  & 0  & 0  & 12      & 0 \\ \hline
1  & 1  & 0  & 1  & 13      & 0 \\ \hline
1  & 1  & 1  & 0  & 14      & 1 \\ \hline
1  & 1  & 1  & 1  & 15      & - \\ \hline
\end{tabular}
\end{table}

\section{MDNK leidmine Karnaugh' kaardiga ja MKNK leidmine McCluskey' meetodiga}
\subsection{McCluskey meetod}
\begin{table}[H]
\centering
\caption{McCluskey kleepetabel}
\label{mccluskey-kleepimine}
\begin{tabular}{|l|l|l|l|}
\hline
indeks & 1. pk & 2-sed interv. & 4-sed \\ \hline\hline
0 & 0000 x & \begin{tabular}[c]{@{}c@{}}000- x\\ 0-00 x\\ -000 x\end{tabular} & \begin{tabular}[c]{@{}c@{}}0-0- A3\\ - -00 A4\end{tabular} \\ \hline
1 & \begin{tabular}[c]{@{}c@{}}0001 x\\ 0100 x\\ 1000 x\end{tabular} & \begin{tabular}[c]{@{}c@{}}0-01 x\\ 010- xx\\ 01-0 x\\ -100 xx\\ 1-00 A1\end{tabular} & \begin{tabular}[c]{@{}c@{}}-10- A5\\ 01- - A6\end{tabular} \\ \hline
2 & \begin{tabular}[c]{@{}c@{}}0101 x\\ 0110 x\\ 1100 x\end{tabular} & \begin{tabular}[c]{@{}c@{}}01-1 x\\ -101 xx\\ 011- x\\ 110- x\end{tabular} & -1-1 A7 \\ \hline
3 & \begin{tabular}[c]{@{}c@{}}0111 x\\ 1011 x\\ 1101 x\end{tabular} & \begin{tabular}[c]{@{}c@{}}-111 x\\ 1-11 A2\\ 11-1 x\end{tabular} &  \\ \hline
4 & 1111 x &  &  \\ \hline
\end{tabular}
\end{table}
Kattetabelist valiti lihtimplikandid A2, A4, A5, A6.
\begin{table}[H]
\centering
\caption{Kattetabel}
\label{kattetabel}
\begin{tabular}{|l|l|l|l|l|l|l|l|l|l|l|l|}
\hline
Lihtimplikant & 0 & 1* & 4* & 5* & 6 & 7 & 8 & 11 & 12 & 13 & 15* \\ \hline
A1            &   &    &    &    &   &   & x &    & x  &    &     \\ \hline
\rowcolor{Goldenrod!30}
A2            &   &    &    &    &   &   &   & x  &    &    & x   \\ \hline
A3            & x & x  & x  & x  &   &   &   &    &    &    &     \\ \hline
\rowcolor{Goldenrod!30}
A4            & x &    & x  &    &   &   & x &    &    &    &     \\ \hline
\rowcolor{Goldenrod!30}
A5            &   &    & x  & x  &   &   &   &    & x  & x  &     \\ \hline
\rowcolor{Goldenrod!30}
A6            &   &    & x  & x  & x & x &   &    &    &    &     \\ \hline
A7            &   &    &    & x  &   & x &   &    &    & x  & x   \\ \hline
\end{tabular}
\end{table}
\begin{align*} 
(\wb{x_1}\lor\wb{x_3}\lor\wb{x_4})\land &&\mbox{A2}\\
(x_3\lor x_4)\land &&\mbox{A4}\\
(\wb{x_2} \lor x_3)\land &&\mbox{A5}\\
(x_1\lor \wb{x_2}) &&\mbox{A6}
\end{align*}
MKNK $(\wb{x_4} \lor \wb{x_3} \lor \wb{x_1}) \land (x_4 \lor x_3) \land (x_3 \lor \wb{x_2}) \land (\wb{x_2} \lor x_1)$

\subsubsection{Karnaugh' kaart McCluskey kontrolliks}
\begin{figure}[H]
\centering
\begin{Karnaugh}
    \contingut{0, -, 1, 1,
    0/-, 0/-, 0, 0,
    0, 1, 1, 0,
    0, 0, 1, 0/-}
   \implicant{15}{11}{yellow}
   \implicant{4}{6}{yellow}
   \implicant{4}{13}{yellow}
   \implicant{0}{8}{yellow}
\end{Karnaugh}
\caption{Karnaugh' kaart laiendatud 0de piirkonnaga}
\label{fig:karnaugh-piirkond0}
\end{figure}
MKNK $(\wb{x_4} \lor \wb{x_3} \lor \wb{x_1}) \land (x_4 \lor x_3) \land (x_3 \lor \wb{x_2}) \land (\wb{x_2} \lor x_1)$
\subsection{Karnaugh' kaardiga MDNK}

\begin{figure}[H]
\centering
\begin{Karnaugh}
    \contingut{0, -/1, 1, 1,
    -, -, 0, 0,
    0, 1, 1, 0,
    0, 0, 1, -}
    \implicantdaltbaix[3pt]{1}{9}{yellow}
   \implicant{14}{10}{yellow}
   \implicant{3}{2}{yellow}
\end{Karnaugh}
\caption{Karnaugh' kaart laiendatud 1de piirkonnaga}
\label{fig:karnaugh-piirkond1}
\end{figure}
MDNK $\wb{x_1}\wb{x_2}x_3 \lor x_1x_3\wb{x_4} \lor \wb{x_2}\wb{x_3}x_4$

% (nx4+nx3+nx1)(x4+x3)(x3+nx2)(nx2+x1)=\\
%=(nx4x3+nx3x4+nx1x4+nx1x3)(x3+nx2)(nx2+x1)=\\
%=(nx4x3+nx4x3nx2+x4nx3nx2+x4x3nx1+x4nx2nx1+x3nx1+x3nx2nx1)(nx2+x1)=\\
%=nx4x3nx2+nx4x3x1+nx4x3nx2+nx4x3nx2x1+x4nx3nx2+x4nx3nx2x1+x4x3nx2nx1+x4nx2nx1+x3nx2nx1+x3nx2nx1=\\
%=nx4x3nx2+nx4x3x1+nx4x3nx2x1+x4nx3nx2+x4nx2nx1+x3nx2nx1=\\
%=nx4x3nx2+nx4x3x1+x4nx3nx2+x4nx2nx1+x3nx2x1

\subsubsection{MDNK ja MKNK loogiline võrdlus}
MDNK $\wb{x_1}\wb{x_2}x_3 \lor x_1x_3\wb{x_4} \lor \wb{x_2}\wb{x_3}x_4$

MKNK $(\wb{x_4} \lor \wb{x_3} \lor \wb{x_1}) \land (x_4 \lor x_3) \land (x_3 \lor \wb{x_2}) \land (\wb{x_2} \lor x_1)$

MDNK ja MKNK ei annaks samasugust tõeväärtustabelit, sest jaotasin DNK ja KNK puhul määramatuspiirkonnad erinevalt.
\section{Disjunktiivse normaalkuju saamine minimaalset konjunktiivset normaalkuju teisendades}
\begin{multline*}
(\wb{x_4} \lor \wb{x_3} \lor \wb{x_1})(x_4 \lor x_3)(x_3 \lor \wb{x_2})(\wb{x_2} \lor x_1)=\\
=(x_3\wb{x_4} \lor \wb{x_3}x_4 \lor \wb{x_1}x_4 \lor \wb{x_1}x_3)(x_3 \lor \wb{x_2})(\wb{x_2} \lor x_1)=\\
=(x_3\wb{x_4} \lor \wb{x_2}x_3\wb{x_4} \lor \wb{x_2}\wb{x_3}x_4 \lor \wb{x_1}x_3x_4 \lor \wb{x_1}\wb{x_2}x_4 \lor \wb{x_1}x_3 \lor \wb{x_1}\wb{x_2}x_3)(\wb{x_2} \lor x_1)=\\
=\Cline[blue]{\wb{x_2}x_3\wb{x_4}} \lor x_1x_3\wb{x_4} \lor \Cline[blue]{\wb{x_2}x_3\wb{x_4}} \lor x_1\wb{x_2}x_3\wb{x_4} \lor \Cline[red]{\wb{x_2}\wb{x_3}x_4} \lor \Cline[red]{x_1\wb{x_2}\wb{x_3}x_4} \lor \Cline[green]{\wb{x_1}\wb{x_2}x_3x_4} \lor \Cline[green]{\wb{x_1}\wb{x_2}x_4} \lor \Cline[yellow]{\wb{x_1}\wb{x_2}x_3} \lor \Cline[yellow]{\wb{x_1}\wb{x_2}x_3}=\\
=\wb{x_2}x_3\wb{x_4} \lor x_1x_3\wb{x_4} \lor x_1\wb{x_2}x_3\wb{x_4} \lor \wb{x_2}\wb{x_3}x_4 \lor \wb{x_1}\wb{x_2}x_4 \lor \wb{x_1}\wb{x_2}x_3=\\
=\wb{x_2}x_3\wb{x_4} \lor x_1x_3\wb{x_4} \lor \wb{x_2}\wb{x_3}x_4 \lor \wb{x_1}\wb{x_2}x_4 \lor x_1\wb{x_2}x_3\\
\end{multline*}
Ei ole kokkulangev avaldis Karnaugh' kaardiga saadud avaldisega.
\subsection{Teisendatud DNK ja Karnaugh' kaardi MDNK loogiline võrdlus}
\begin{table}[H]
\centering
\caption{Teisendatud DNK tõeväärtustabel}
\label{my-label}
\begin{tabular}{|c|l|l|l|l||l|}
\hline
arv & $x_1$ & $x_2$ & $x_3$ & $x_4$ & $\wb{x_2}x_3\wb{x_4} \lor x_1x_3\wb{x_4} \lor \wb{x_2}\wb{x_3}x_4 \lor \wb{x_1}\wb{x_2}x_4 \lor x_1\wb{x_2}x_3$ \\ \hline
15 & 1 & 1 & 1 & 1 & 1                                                                     \\ \hline
14 & 1 & 1 & 1 & 0 & 1                                                                     \\ \hline
13 & 1 & 1 & 0 & 1 & 1                                                                     \\ \hline
12 & 1 & 1 & 0 & 0 & 0                                                                     \\ \hline
11 & 1 & 0 & 1 & 1 & 0                                                                     \\ \hline
10 & 1 & 0 & 1 & 0 & 1                                                                     \\ \hline
9 & 1 & 0 & 0 & 1 & 0                                                                     \\ \hline
8 & 1 & 0 & 0 & 0 & 0                                                                     \\ \hline
7 & 0 & 1 & 1 & 1 & 1                                                                     \\ \hline
6 & 0 & 1 & 1 & 0 & 1                                                                     \\ \hline
5 & 0 & 1 & 0 & 1 & 0                                                                     \\ \hline
4 & 0 & 1 & 0 & 0 & 0                                                                     \\ \hline
3 & 0 & 0 & 1 & 1 & 0                                                                     \\ \hline
2 & 0 & 0 & 1 & 0 & 0                                                                     \\ \hline
1 & 0 & 0 & 0 & 1 & 0                                                                     \\ \hline
0 & 0 & 0 & 0 & 0 & 0                                                                     \\ \hline
\end{tabular}
\end{table}
\begin{table}[H]
\centering
\caption{Karnaugh' kaardiga saadud MDNK tõeväärtustabel}
\label{my-label}
\begin{tabular}{|c|l|l|l|l||l|}
\hline
arv & $x_1$ & $x_2$ & $x_3$ & $x_4$ & $\wb{x_1}\wb{x_2}x_3 \lor x_1x_3\wb{x_4} \lor \wb{x_2}\wb{x_3}x_4$ \\\hline
15 & 1 & 1 & 1 & 1 & 1                                         \\ \hline
14 & 1 & 1 & 1 & 0 & 0                                         \\ \hline
13 & 1 & 1 & 0 & 1 & 0                                         \\ \hline
12 & 1 & 1 & 0 & 0 & 0                                         \\ \hline
11 & 1 & 0 & 1 & 1 & 1                                         \\ \hline
10 & 1 & 0 & 1 & 0 & 0                                         \\ \hline
9 & 1 & 0 & 0 & 1 & 0                                         \\ \hline
8 & 1 & 0 & 0 & 0 & 0                                         \\ \hline
7 & 0 & 1 & 1 & 1 & 1                                         \\ \hline
6 & 0 & 1 & 1 & 0 & 1                                         \\ \hline
5 & 0 & 1 & 0 & 1 & 0                                         \\ \hline
4 & 0 & 1 & 0 & 0 & 0                                         \\ \hline
3 & 0 & 0 & 1 & 1 & 1                                         \\ \hline
2 & 0 & 0 & 1 & 0 & 0                                         \\ \hline
1 & 0 & 0 & 0 & 1 & 0                                         \\ \hline
0 & 0 & 0 & 0 & 0 & 0                                         \\ \hline
\end{tabular}
\end{table}
Teisendatud ja Karnaugh' kaardiga leitud normaalkujud ei ole loogiliselt võrdsed, sest jaotasin DNK ja KNK puhul määramatuspiirkonnad erinevalt. Ühe avaldise sain KNK määramatuste jaotusega (teisendamisel) ja teise 1-de piirkonna Karnaugh' kaardi määramatuste jaotusega.

\section{Taandatud DNK ja Täielik DNK}
\subsection{Taandatud DNK}
Lisan varem leitud MDNK lihtimplikantidele veel puuduolevad lihtimplikandid. Nende leidmiseks vaatan MDNK leidmiseks kasutatud Karnaugh' kaarti ning märgin need sellel.

\begin{figure}[H]
\centering
\begin{Karnaugh}
    \contingut{0, -/1, 1, 1,
    -, -, 0, 0,
    0, 1, 1, 0,
    0, 0, 1, -}
    \implicantdaltbaix[3pt]{1}{9}{Melon}
   \implicant{14}{10}{Melon}
   \implicant{3}{2}{Melon}
\end{Karnaugh}
\caption{MDNK leidmiseks kasutatud Karnaugh' kaart}
\label{fig:karnaugh-piirkond1}
\end{figure}

\begin{figure}[H]
\centering
\begin{Karnaugh}
    \contingut{0, -/1, 1, 1,
    -, -, 0, 0,
    0, 1, 1, 0,
    0, 0, 1, -}
   \implicant{1}{3}{Lavender}
   \implicantdaltbaix[3pt]{2}{10}{Lavender}
\end{Karnaugh}
\caption{MDNK Karnaugh' kaardil puuduolevad lihtimplikandid}
\label{fig:karnaugh-piirkond1}
\end{figure}
Puudu olnud liikmed: $\wb{x_1}\wb{x_2}x_4 \lor \wb{x_2}x_3\wb{x_4}$\\
Taandatud DNK: $\wb{x_1}\wb{x_2}x_3 \lor x_1x_3\wb{x_4} \lor \wb{x_2}\wb{x_3}x_4 \lor \wb{x_1}\wb{x_2}x_4 \lor \wb{x_2}x_3\wb{x_4}$
\subsection{Täielik DNK}
Leian täieliku DNK, kasutades DNK laiendatud 1-de piirkonna tõeväärtustabelit:
\begin{table}[H]
\centering
\caption{laiendatud 1-de piirkonna tõeväärtustabel}
\label{truth-table-wide}
\begin{tabular}{|c|c|c|c|c||c|}
\hline
$x_1$ & $x_2$ & $x_3$ & $x_4$ & 10. arv & F \\ \hline\hline
0  & 0  & 0  & 0  & 0       & 0 \\ \hline
0  & 0  & 0  & 1  & 1       & 1 \\ \hline
0  & 0  & 1  & 0  & 2       & 1 \\ \hline
0  & 0  & 1  & 1  & 3       & 1 \\ \hline
0  & 1  & 0  & 0  & 4       & - \\ \hline
0  & 1  & 0  & 1  & 5       & - \\ \hline
0  & 1  & 1  & 0  & 6       & 0 \\ \hline
0  & 1  & 1  & 1  & 7       & 0 \\ \hline
1  & 0  & 0  & 0  & 8       & 0 \\ \hline
1  & 0  & 0  & 1  & 9       & 1 \\ \hline
1  & 0  & 1  & 0  & 10      & 1 \\ \hline
1  & 0  & 1  & 1  & 11      & 0 \\ \hline
1  & 1  & 0  & 0  & 12      & 0 \\ \hline
1  & 1  & 0  & 1  & 13      & 0 \\ \hline
1  & 1  & 1  & 0  & 14      & 1 \\ \hline
1  & 1  & 1  & 1  & 15      & - \\ \hline
\end{tabular}
\end{table}
Funktsiooni väärtustest 1 saan argumentvektorid: 
\[x_1\wb{x_2}\wb{x_3}\wb{x_4} \lor \wb{x_1}x_2\wb{x_3}\wb{x_4} \lor x_1x_2\wb{x_3}\wb{x_4} \lor x_1\wb{x_2}\wb{x_3}x_4 \lor \wb{x_1}x_2\wb{x_3}x_4 \lor \wb{x_1}x_2x_3x_4\]

\section{MKNK'ga (loogiliselt) võrdne taandatud KNK ja täielik KNK}
\subsection{MKNK'ga (loogiliselt) võrdne Täielik KNK}
\begin{table}[H]
\centering
\caption{laiendatud 1-de piirkonna tõeväärtustabel}
\label{truth-table-wide}
\begin{tabular}{|c|c|c|c|c||c|}
\hline
$x_1$ & $x_2$ & $x_3$ & $x_4$ & 10. arv & F \\ \hline\hline
0  & 0  & 0  & 0  & 0       & 0 \\ \hline
0  & 0  & 0  & 1  & 1       & - \\ \hline
0  & 0  & 1  & 0  & 2       & 1 \\ \hline
0  & 0  & 1  & 1  & 3       & 1 \\ \hline
0  & 1  & 0  & 0  & 4       & 0 \\ \hline
0  & 1  & 0  & 1  & 5       & 0 \\ \hline
0  & 1  & 1  & 0  & 6       & 0 \\ \hline
0  & 1  & 1  & 1  & 7       & 0 \\ \hline
1  & 0  & 0  & 0  & 8       & 0 \\ \hline
1  & 0  & 0  & 1  & 9       & 1 \\ \hline
1  & 0  & 1  & 0  & 10      & 1 \\ \hline
1  & 0  & 1  & 1  & 11      & 0 \\ \hline
1  & 1  & 0  & 0  & 12      & 0 \\ \hline
1  & 1  & 0  & 1  & 13      & 0 \\ \hline
1  & 1  & 1  & 0  & 14      & 1 \\ \hline
1  & 1  & 1  & 1  & 15      & 0 \\ \hline
\end{tabular}
\end{table}
Saan Täieliku KNK:
\begin{multline*}(x_1 \lor x_2 \lor x_3 \lor x_4)(x_1 \lor \wb{x_2} \lor x_3 \lor x_4)(x_1 \lor \wb{x_2} \lor x_3 \lor \wb{x_4})(x_1 \lor \wb{x_2} \lor \wb{x_3} \lor x_4)(x_1 \lor \wb{x_2} \lor \wb{x_3} \lor \wb{x_4}) \land\\
\land (\wb{x_1} \lor x_2 \lor x_3 \lor x_4)(\wb{x_1} \lor x_2 \lor \wb{x_3} \lor \wb{x_4}) \land (\wb{x_1} \lor \wb{x_2} \lor x_3 \lor x_4)(\wb{x_1} \lor \wb{x_2} \lor x_3 \lor \wb{x_4})(\wb{x_1} \lor \wb{x_2} \lor \wb{x_3} \lor \wb{x_4})\\
\end{multline*}

\section{Shannoni disjunktiivne arendus enim esinenud argumendi järgi}
Leian enim esinenud argumendi MDNKs $\wb{x_1}\wb{x_2}x_3 \lor x_1x_3\wb{x_4} \lor \wb{x_2}\wb{x_3}x_4$
\begin{table}[H]
\centering
\caption{Argumentide kordsus MDNKs}
\label{my-label}
\begin{tabular}{|c|l|l|l|l||l|}
\hline
argument & argumendi kordsus MDNKs \\ \hline
$x_1$ & 2                                                                     \\ \hline
$x_2$ & 2                                                                     \\ \hline
$x_3$ & 3                                                                     \\ \hline
$x_4$ & 2                                                                     \\ \hline
\end{tabular}
\end{table}
Enim esines MDNKs argumenti $x_3$, teen selle järgi Shannoni disjunktiivse arenduse:
\begin{multline*}
f=\wb{x_3} \cdot f(x_1x_2 0 x_4) \lor x_3 \cdot f(x_1x_2 1 x_4)=\\
=\wb{x_3}(\wb{x_1}\wb{x_2} \cdot 0 \lor x_1 \cdot 0 \cdot \wb{x_4} \lor \wb{x_2} \cdot \wb{0} \cdot x_4) \lor x_3(\wb{x_1}\wb{x_2} \cdot 1 \lor x_1 \cdot 1 \cdot \wb{x_4} \lor \wb{x_2} \cdot \wb{1} \cdot x_4)=\\
=\wb{x_3}(\wb{x_2}x_4) \lor x_3(\wb{x_1}\wb{x_2} \lor x_1\wb{x_4})\\
\end{multline*}
\section{Shannoni disjunktiivne arendus $x_1$ ja $x_2$ järgi}
\begin{multline*}
f=\widebar{x_1}\widebar{x_2} \cdot f(00 x_3 x_4) \lor \widebar{x_1}x_2 \cdot f(01 x_3 x_4) \lor x_1\widebar{x_2} \cdot f(10 x_3 x_4) \lor x_1x_2 \cdot f(11 x_3 x_4)=\\
=\widebar{x_1}\widebar{x_2} \cdot (\widebar{0} \cdot \widebar{0} \cdot x_3 \lor 0 \cdot x_3\widebar{x_4} \lor \widebar{0} \cdot \widebar{x_3}x_4) \lor \\ \lor \widebar{x_1}x_2(\widebar{0} \cdot \widebar{1} \cdot x_3 \lor 0 \cdot x_3\widebar{x_4} \lor \widebar{1} \cdot \widebar{x_3}x_4) \lor \\ \lor x_1\widebar{x_2}(\widebar{1} \cdot \widebar{0} \cdot x_3 \lor 1 \cdot x_3\widebar{x_4} \lor \widebar{0} \cdot \widebar{x_3}x_4) \lor \\ \lor x_1x_2(\widebar{1}\widebar{1} \cdot x_3 \lor 1 \cdot x_3\widebar{x_4} \lor \widebar{1} \cdot \widebar{x_3}x_4)=\\
=\widebar{x_1}\widebar{x_2}(x_3 \lor \widebar{x_3}x_4) \lor \widebar{x_1}x_2 \cdot 0 \lor x_1\widebar{x_2}(x_3\widebar{x_4} \lor \widebar{x_3}x_4) \lor x_1x_2(x_3\widebar{x_4})=\\
=\widebar{x_1}\widebar{x_2}(x_3 \lor \cancel{\widebar{x_3}}x_4) \lor x_1\widebar{x_2}(x_3\widebar{x_4} \lor \widebar{x_3}x_4) \lor x_1x_2(x_3\widebar{x_4})\\
\end{multline*}
\section{Shannoni konjunktiivne arendus $x_1$ ja $x_2$ järgi}
\begin{multline*}
    f=(x_1 \lor x_2 \lor f(00 x_3 x_4))(x_1 \lor \widebar{x_2} \lor f(01 x_3x_4))(\widebar{x_1} \lor x_2 \lor f(10 x_3 x_4))(\widebar{x_1} \lor \widebar{x_2} \lor f(11 x_3 x_4))=\\
=(x_1 \lor x_2 \lor (\widebar{0} \cdot \widebar{0} \cdot x_3 \lor 0 \cdot x_3\widebar{x_4} \lor \widebar{0} \cdot \widebar{x_3}x_4))(x_1 \lor \widebar{x_2} \lor (\widebar{0} \cdot \widebar{1} \cdot x_3 \lor 0 \cdot x_3\widebar{x_4} \lor \widebar{1} \cdot \widebar{x_3}x_4)) \land \\
\land (\widebar{x_1} \lor x_2 \lor (\widebar{1} \cdot \widebar{0} \cdot x_3 \lor 1 \cdot x_3\widebar{x_4} \lor \widebar{0} \cdot \widebar{x_3}x_4))(\widebar{x_1} \lor \widebar{x_2} \lor (\widebar{1}\widebar{1} \cdot x_3 \lor 1 \cdot x_3\widebar{x_4} \lor \widebar{1} \cdot \widebar{x_3}x_4)))=\\
=(x_1 \lor x_2 \lor (x_3 \lor \widebar{x_3}x_4))(x_1 \lor \widebar{x_2} \lor (0)) \land (x_3\widebar{x_4} \lor \widebar{x_3}x_4))(\widebar{x_1} \lor \widebar{x_2} \lor (x_3\widebar{x_4})))=\\
=(x_1 \lor x_2 \lor x_3 \lor \cancel{\widebar{x_3}}x_4)(x_1 \lor \widebar{x_2})(x_1\lor\wb{x_2}\lor x_3\widebar{x_4} \lor \widebar{x_3}x_4)(\widebar{x_1} \lor \widebar{x_2} \lor x_3\widebar{x_4})\\
\end{multline*}
\section{MDNK tuletis}
\subsection{MDNK tuletis muutuja $x_2$ järgi}
\begin{multline*}
    f(x_1 0 x_3 x_4) \oplus f(x_1 1 x_3 x_4)= (\wb{x_1}x_3 \lor x_1x_3\wb{x_4} \lor \wb{x_3}x_4) \oplus x_1x_3\wb{x_4}=\\
=\overline{(\wb{x_1}x_3 \lor x_1x_3\wb{x_4} \lor \wb{x_3}x_4)}x_1x_3\wb{x_4} \lor \overline{x_1x_3\wb{x_4}}(\wb{x_1}x_3 \lor x_1x_3\wb{x_4} \lor \wb{x_3}x_4)=\\
=(\overline{\wb{x_1}x_3} \land \overline{x_1x_3\wb{x_4}} \land \overline{\wb{x_3}x_4})x_1x_3\wb{x_4} \lor (\wb{x_1} \lor \wb{x_3} \lor x_4)(\wb{x_1}x_3 \lor x_1x_3\wb{x_4} \lor \wb{x_3}x_4)=\\
=(x_1 \lor \wb{x_3})(\wb{x_1} \lor \wb{x_3} \lor x_4)(x_3 \lor \wb{x_4})x_1x_3\wb{x_4} \lor (\wb{x_1}x_3 \lor \wb{x_1}\wb{x_3}x_4 \lor \wb{x_3}x_4 \lor \wb{x_1}x_3x_4 \lor \cancel{x_3x_4})=\\
=(x_1\wb{x_3} \lor x_1x_4 \lor \wb{x_1}\wb{x_3} \lor \wb{x_3} \lor \wb{x_3}x_4)(x_3 \lor \wb{x_4})x_1x_3\wb{x_4} \lor (\wb{x_1}x_3 \lor \wb{x_1}\wb{x_3}x_4 \lor \wb{x_3}x_4 \lor \wb{x_1}x_3x_4)=\\
=\wb{x_1}\tikzmark{a}x_3 \lor \overwrite{\cancel{\wb{x_1}\wb{x_3}x_4} \lor \wb{x_3}x_4}{neeldumine} \lor \cancel{\wb{x_1}x_3x_4}\tikzmark{b}=\wb{x_1}x_3 \lor \wb{x_3}x_4\\
  \begin{tikzpicture}[overlay,remember picture,out=315,in=225,distance=0.7cm]
  \def\myshift#1{\raisebox{-2.5ex}}
    \draw[->,BurntOrange,shorten >=4pt,shorten <=4pt,postaction={decorate,decoration={text along path,text align=center,text={|\myshift|neeldumine}}}] (a.center) to  (b.center);
  \end{tikzpicture}
\end{multline*}
\subsection{MDNK tuletis muutuja $x_4$ järgi}
\begin{multline*}
    f(x_1 x_2 x_3 0) \oplus f(x_1x_2x_3 1)=(\wb{x_1}\wb{x_2}x_3 \lor x_1x_3) \oplus (\wb{x_1}\wb{x_2}x_3 \lor \wb{x_2}x_3)=\\
    =\overline{(\wb{x_1}\wb{x_2}x_3 \lor x_1x_3)}(\wb{x_1}\wb{x_2}x_3 \lor \wb{x_2}x_3) \lor (\wb{x_1}\wb{x_2}x_3 \lor x_1x_3)\overline{(\wb{x_1}\wb{x_2}x_3 \lor \wb{x_2}x_3)}\\
\end{multline*}

\section{Reed-Mulleri polünoom}
\begin{multline*}
    \widebar{x_1}\widebar{x_2}x_3 \lor x_1x_3\widebar{x_4} \lor \widebar{x_2}\widebar{x_3}x_4=\widebar{x_1}\widebar{x_2}x_3 \oplus x_1x_3\widebar{x_4} \oplus \widebar{x_2}\widebar{x_3}x_4=\\
=(x_1 \oplus 1)(x_2 \oplus 1)x_3 \oplus x_1x_3(x_4 \oplus 1) \oplus (x_2 \oplus 1)(x_3 \oplus 1)x_4=\\
=x_1x_2x_3 \oplus \cancel{x_1x_3} \oplus x_2x_3 \oplus x_3 \oplus x_1x_3x_4 \oplus \cancel{x_1x_3} \oplus x_2x_3x_4 \oplus x_2x_4 \oplus x_3x_4 \oplus x_4\\
\end{multline*}
\end{document}
