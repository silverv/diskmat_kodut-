\documentclass{article}
\usepackage[utf8]{inputenc}
\usepackage[estonian]{babel}
\usepackage{float}
\usepackage{amsmath}
\usepackage{xcolor, colortbl}
\definecolor{LightCyan}{rgb}{0.88,1,1}
\newcolumntype{g}{>{\columncolor{LightCyan}}c}
% Widebar is now a replacement for widebar
\DeclareFontFamily{U}{mathx}{\hyphenchar\font45}
\DeclareFontShape{U}{mathx}{m}{n}{ <-> mathx10 }{}
\DeclareSymbolFont{mathx}{U}{mathx}{m}{n}
\DeclareFontSubstitution{U}{mathx}{m}{n}
\DeclareMathAccent{\widebar}{\mathalpha}{mathx}{"73}

\makeatletter
\newcommand{\cwidebar}[2][0]{{\mathpalette\@cwidebar{{#1}{#2}}}}
\newcommand{\@cwidebar}[2]{\@cwideb@r{#1}#2}
\newcommand{\@cwideb@r}[3]{%
  \sbox\z@{$\m@th#1\mkern-#2mu#3\mkern#2mu$}%
  \widebar{\box\z@}%
}
\makeatother

\title{Diskreetse matemaatika kodutöö}
\author{Silver Valdvee}
\date{detsember 2017}
\input{karnaugh.tex}
\begin{document}
\maketitle

\section{Funktsiooni tekitamine}
\begin{enumerate}
    \item Martiklinumber: 179390
    \item HEX: 2BCBE
    \item Seitsmega korrutades esimene seitsmekohaline 16. arv: 3AAE292
    \item Unikaalsed järguväärtused määravad 1-de piirkonna: 3, A, E, 2, 9
    \item Unikaalsed järguväärtused 10. süsteemis 1-de piirkond: 3, 10, 14, 2, 9
    \item Eelkirjeldatud viisil saadud ja hetkel kalkulaatoris näidatava 16. arvu korrutis 7 kuni 9-kohalise arvu saamiseni: $4E9F5919E$
    \item 9-kohalise tekkinud 16ndarvu need järguväärtused 0 ... 15, mis ei kuulu juba 1-de piirkonda, moodustavad funktsiooni määramatuspiirkonna: $4, F, 5, 1 \rightarrow 4, 15, 5, 1 \rightarrow 1, 4, 5, 15$
    \item Ülejäänud arvud vahemikus 0 ...15 (mis puuduvad nii 1de piirkonnas kui ka määramatuspiirkonnas) moodustavad 0de piirkonna: 0, 6, 7, 8, 11, 12, 13
\end{enumerate}
Saan funktsiooni:
\[f(x_1...x_4)=\Sigma(2,3,9,10,14)_1(0,6,7,8,11,12,13)_0(1,4,5,15)_{\_}
\]
Teen sellele tõeväärtustabeli:
\begin{table}[H]
\centering
\caption{tõeväärtustabel funktsioonile}
\label{truth-table}
\begin{tabular}{|c|c|c|c|c||g|}
\hline
$x_4$ & $x_3$ & $x_2$ & $x_1$ & 10. arv & F \\ \hline\hline
0  & 0  & 0  & 0  & 0       & 0 \\ \hline
0  & 0  & 0  & 1  & 1       & - \\ \hline
0  & 0  & 1  & 0  & 2       & 1 \\ \hline
0  & 0  & 1  & 1  & 3       & 1 \\ \hline
0  & 1  & 0  & 0  & 4       & - \\ \hline
0  & 1  & 0  & 1  & 5       & - \\ \hline
0  & 1  & 1  & 0  & 6       & 0 \\ \hline
0  & 1  & 1  & 1  & 7       & 0 \\ \hline
1  & 0  & 0  & 0  & 8       & 0 \\ \hline
1  & 0  & 0  & 1  & 9       & 1 \\ \hline
1  & 0  & 1  & 0  & 10      & 1 \\ \hline
1  & 0  & 1  & 1  & 11      & 0 \\ \hline
1  & 1  & 0  & 0  & 12      & 0 \\ \hline
1  & 1  & 0  & 1  & 13      & 0 \\ \hline
1  & 1  & 1  & 0  & 14      & 1 \\ \hline
1  & 1  & 1  & 1  & 15      & - \\ \hline
\end{tabular}
\end{table}
McCluskey meetod:
\begin{table}[H]
\centering
\caption{McCluskey meetod}
\label{mccluskey-kleepimine}
\begin{tabular}{|l|l|l|l|}
\hline
indeks & 1. pk & 2-sed interv. & 4-sed \\ \hline\hline
0 & 0000 x & \begin{tabular}[c]{@{}c@{}}000- x\\ 0-00 x\\ -000 x\end{tabular} & \begin{tabular}[c]{@{}c@{}}0-0- A3\\ - -00 A4\end{tabular} \\ \hline
1 & \begin{tabular}[c]{@{}c@{}}0001 x\\ 0100 x\\ 1000 x\end{tabular} & \begin{tabular}[c]{@{}c@{}}0-01 x\\ 010- xx\\ 01-0 x\\ -100 xx\\ 1-00 A1\end{tabular} & \begin{tabular}[c]{@{}c@{}}-10- A5\\ 01- - A6\end{tabular} \\ \hline
2 & \begin{tabular}[c]{@{}c@{}}0101 x\\ 0110 x\\ 1100 x\end{tabular} & \begin{tabular}[c]{@{}c@{}}01-1 x\\ -101 xx\\ 011- x\\ 110- x\end{tabular} & -1-1 A7 \\ \hline
3 & \begin{tabular}[c]{@{}c@{}}0111 x\\ 1011 x\\ 1101 x\end{tabular} & \begin{tabular}[c]{@{}c@{}}-111 x\\ 1-11 A2\\ 11-1 x\end{tabular} &  \\ \hline
4 & 1111 x &  &  \\ \hline
\end{tabular}
\end{table}
Kattetabelist valiti lihtimplikandid A2, A4, A5, A6.
\begin{table}[H]
\centering
\caption{Kattetabel}
\label{kattetabel}
\begin{tabular}{|l|l|l|l|l|l|l|l|l|l|l|l|}
\hline
Lihtimplikant & 0 & 1* & 4* & 5* & 6 & 7 & 8 & 11 & 12 & 13 & 15* \\ \hline
A1            &   &    &    &    &   &   & x &    & x  &    &     \\ \hline
\rowcolor{LightCyan}
A2            &   &    &    &    &   &   &   & x  &    &    & x   \\ \hline
A3            & x & x  & x  & x  &   &   &   &    &    &    &     \\ \hline
\rowcolor{LightCyan}
A4            & x &    & x  &    &   &   & x &    &    &    &     \\ \hline
\rowcolor{LightCyan}
A5            &   &    & x  & x  &   &   &   &    & x  & x  &     \\ \hline
\rowcolor{LightCyan}
A6            &   &    & x  & x  & x & x &   &    &    &    &     \\ \hline
A7            &   &    &    & x  &   & x &   &    &    & x  & x   \\ \hline
\end{tabular}
\end{table}

\begin{align*} 
(\widebar{x_1}\lor\widebar{x_3}\lor\widebar{x_4})\land &&\mbox{A2}\\
(x_3\lor x_4)\land &&\mbox{A4}\\
(\widebar{x_2} \lor x_3)\land &&\mbox{A5}\\
(x_1\lor \widebar{x_2}) &&\mbox{A6}
\end{align*}

MKNK $(\widebar{x_4} \lor \widebar{x_3} \lor \widebar{x_1}) \land (x_4 \lor x_3) \land (x_3 \lor \widebar{x_2}) \land (\widebar{x_2} \lor x_1)$

\begin{figure}[H]
\centering
\begin{Karnaugh}
    \contingut{0, -, 1, 1,
    0/-, 0/-, 0, 0,
    0, 1, 1, 0,
    0, 0, 1, 0/-}
   \implicant{15}{11}{yellow}
   \implicant{4}{6}{yellow}
   \implicant{4}{13}{yellow}
   \implicant{0}{8}{yellow}
\end{Karnaugh}
\caption{Karnaugh kaart laiendatud 0de piirkonnaga}
\label{fig:karnaugh-piirkond0}
\end{figure}




\begin{figure}[H]
\centering
\begin{Karnaugh}
    \contingut{0, -/1, 1, 1,
    -, -, 0, 0,
    0, 1, 1, 0,
    0, 0, 1, -}
    \implicantdaltbaix[3pt]{1}{9}{yellow}
   \implicant{14}{10}{yellow}
   \implicant{3}{2}{yellow}
\end{Karnaugh}
\caption{Karnaugh kaart laiendatud 1de piirkonnaga}
\label{fig:karnaugh-piirkond1}
\end{figure}
MDNK $\widebar{x_1}\widebar{x_2}x_3 \lor x_1x_3\widebar{x_4} \lor \widebar{x_2}\widebar{x_3}x_4$
% (nx4+nx3+nx1)(x4+x3)(x3+nx2)(nx2+x1)=(nx4x3+nx3x4+nx1x4+nx1x3)(x3+nx2)(nx2+x1)=(nx4x3+nx4x3nx2+x4nx3nx2+x4x3nx1+x4nx2nx1+x3nx1+x3nx2nx1)(nx2+x1)=nx4x3nx2+nx4x3x1+nx4x3nx2+nx4x3nx2x1+x4nx3nx2+x4nx3nx2x1+x4x3nx2nx1+x4nx2nx1+x3nx2nx1+x3nx2nx1=
\begin{multline*}
(\widebar{x_4} \lor \widebar{x_3} \lor \widebar{x_1}) \land (x_4 \lor x_3) \land (x_3 \lor \widebar{x_2}) \land (\widebar{x_2} \lor x_1)=\\
=(\widebar{x_4}x_3 \lor \widebar{x_3}x_4 \lor \widebar{x_1}x_4 \lor \widebar{x_1}x_3) \land (x_3 \lor \widebar{x_2}) \land (\widebar{x_2} \lor x_1)=\\
=(\widebar{x_4}x_3 \lor \widebar{x_4}x_3\widebar{x_2} \lor x_4\widebar{x_3}\widebar{x_2} \lor x_4x_3\widebar{x_1} \lor x_4\widebar{x_2}\widebar{x_1} \lor x_3\widebar{x_1} \lor x_3\widebar{x_2}\widebar{x_1}) \land (\widebar{x_2} \lor x_1)=\\
=\widebar{x_4}x_3\widebar{x_2} \lor \widebar{x_4}x_3x_1 \lor \widebar{x_4}x_3\widebar{x_2} \lor \widebar{x_4}x_3\widebar{x_2}x_1 \lor x_4\widebar{x_3}\widebar{x_2} \lor x_4\widebar{x_3}\widebar{x_2}x_1 \lor x_4x_3\widebar{x_2}\widebar{x_1} \lor x_4\widebar{x_2}\widebar{x_1} \lor x_3\widebar{x_2}\widebar{x_1} \lor x_3\widebar{x_2}\widebar{x_1}=\\
=\widebar{x_4}x_3\widebar{x_2} \lor \widebar{x_4}x_3x_1 \lor \widebar{x_4}x_3\widebar{x_2}x_1 \lor x_4\widebar{x_3}\widebar{x_2} \lor x_4\widebar{x_2}\widebar{x_1} \lor x_3\widebar{x_2}\widebar{x_1}=\\
=\widebar{x_4}x_3\widebar{x_2} \lor \widebar{x_4}x_3x_1 \lor x_4\widebar{x_3}\widebar{x_2} \lor x_4\widebar{x_2}\widebar{x_1} \lor x_3\widebar{x_2}x_1
\end{multline*}

\end{document}
